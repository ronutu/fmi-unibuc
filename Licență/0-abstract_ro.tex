\begin{abstractpage}
\begin{abstract}{romanian}
	
Educația este esențială într-o societate și este important ca manualele școlare să aibă o calitate cât mai ridicată. În fiecare an, Ministerul Educației organizează licitații cu scopul de a achiziționa manuale cu un raport calitate-preț optim. Pe lângă variantele tipărite, Ministerul propune și crearea variantelor digitale ale acestor manuale.

Două dintre avantajele manualelor digitale sunt posibilitatea de a fi accesate de oriunde, precum și reducerea volumului de cărți pe care elevii îl transportă zilnic la școală.

Editura Intuitext are experiență de mai mulți ani în crearea manualelor și caută să își optimizeze procesul de creare în fiecare an. Până acum, transformarea manualelor în format digital era un proces care necesita foarte mult timp, atenție și resurse umane. Fiind un proces manual, riscul de erori era ridicat.

În această lucrare este descrisă  o aplicație de automatizare prin care se optimizează conversia manualelor tipărite în format digital, având ca obiectiv obținerea unui procent de similaritate cât mai ridicat. Acest proces eficient contribuie indirect la creșterea profitabilității editurii Intuitext.

\end{abstract}
\end{abstractpage}






